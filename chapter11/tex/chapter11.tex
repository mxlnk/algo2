\documentclass[8pt, DIV15, twocolumn]{scrartcl}
\usepackage[utf8]{inputenc}
\usepackage[T1]{fontenc}
\usepackage{amsfonts}
\usepackage[german]{babel}
\usepackage{amsmath}
\usepackage{caption}
\usepackage{float} 
\usepackage{color}
\usepackage{bm}
\usepackage{listings}

\usepackage{graphicx}

\usepackage{ae}
\subject{\vspace{-1\baselineskip}}

\title{Chapter 11 - Unterteilungsalgorithmen für Flächen}
\date{}

\publishers{\vspace{-.5\baselineskip}}

\begin{document}
\setlength{\abovedisplayskip}{0pt}
\setlength{\belowdisplayskip}{0pt}
\setlength{\parskip}{0pt}
\setlength{\topmargin}{0pt}

 
\maketitle

\thispagestyle{empty}

\section*{Allgemein}
Regelmäßiges biinfinites Kontrollnetz $C$ und das unterteilte Netz $B$ mit den Unterteilungsmatrizen $U$ und $V$:

\begin{equation*}
\begin{aligned}
C &:= [\mathbf{c}_{ij}]_{i,j\in \mathbb{Z}} = [\mathbf{c_i}]_{\mathbf{i}\in\mathbb{Z}^2} =: \mathbf{c}_{\mathbb{Z}^2} \\
B &:= \mathbf{b}_{\mathbb{Z}^2} := UCV^t \\
&\left(U,V\right) \text{ heißt Tempus}
\end{aligned}
\end{equation*}

Bem.: Wenn $U, V$ konvergente Kurvenunterteilungsalgorithmen, dann konvergiert $U^k C \left(V^t\right)^k$ gegen eine Fläche.

\section*{Symbole}

\begin{equation*}
\begin{aligned}
C = \mathbf{c}_{\mathbb{Z}^2} &\Rightarrow \mathbf{c} \left(\mathbf{x}\right) := \mathbf{c} \left(x,y\right) := \sum\limits_{i\in\mathbb{Z}} \sum\limits_{j\in\mathbb{Z}} \mathbf{c}_{ij} x^i y^j := \sum\limits_{\mathbf{i}\in\mathbb{Z}^2} \mathbf{c_i x^i} \\
B := UCV^t &\Rightarrow \mathbf{b} \left(x,y\right) := \alpha \left(x\right) \mathbf{c} \left(x^2,y^2\right) \beta \left(y\right) \\
\left(U,V\right) &\Rightarrow \gamma \left(x,y\right) := \alpha\left(x\right) \beta\left(y\right) \text{ (Symbol des Tempus)}
\end{aligned}
\end{equation*}

Unterteilungsgleichung: $\mathbf{b}\left(\mathbf{x}\right) = \gamma \left(\mathbf{x}\right) \mathbf{c} \left(\mathbf{x}^2\right)$ bzw. komponentenweise: $\mathbf{b_i} = \sum\limits_{\mathbf{j}\in\mathbb{Z}^2} \gamma_{\mathbf{i}-2\mathbf{j}} \mathbf{c_j}$

\section*{Masken}

Wenn $\Gamma := \gamma_{\mathbb{Z}^2}$ biinfinite Matrix. Also

\begin{equation*}
\begin{aligned}
&\gamma\left(x,y\right) := [...x^{-1} x^0 x^1 ...] \Gamma [... y^{-1} y^0 y^1 ...]^t \\
&\Gamma_{\mathbf{i}} := [\gamma_{\mathbf{i}-2\mathbf{j}}]_{\mathbf{j}\in\mathbb{Z}^2} \text{heißen Masken für } \mathbf{i} = (0,0), (1,0), (0,1), (1,1)
\end{aligned} 
\end{equation*}



\end{document}