\documentclass[]{scrartcl}
\usepackage[utf8]{inputenc}
\usepackage{amsmath}


\subject{\vspace{-1\baselineskip}}



\begin{document}
\setlength{\abovedisplayskip}{0pt}
\setlength{\belowdisplayskip}{0pt}
\setlength{\parskip}{0pt}
\setlength{\topmargin}{0pt}

\thispagestyle{empty}

\section{Gradientenverfahren}
Ausgehend vom Anfangspunkt  $ {\bf x_{0}}  =  [x_{1},..,x_{n}] $, berechne Gradient der Kostenfunktion an dieser Stelle: $ \bigtriangledown c({\bf x_{0}}) = [ \partial c / \partial x_{1} , ... , \partial c / \partial x_{n} ] $. Der Nachfolger ist dann, $ {\bf x_{i+1}} = {\bf x_{i}} + h * \bigtriangledown c({\bf x_{i}}) $, wobei $ h > 0 $, falls maximiert werden soll und $ h < 0 $, wenn minimiert wird.

\section{Suchverfahren}
Suchbaum/Graph

\section{stochastische verfahren}
\subsection{Allgemein}
startzustand \textbf{q}, einen nachbar \textbf{p} zufällig wählen, schritt von ausgang zum neuen nachbar akzeptabel?, falls nein neuer nachbar, sonst von akzeptiertem zustand weiter, bis zurfieden

\subsection{simuliertes tempern}
Hier: Maximalzahl betrachteter nachbarn insgesamt (Bedingung fuer Terminieren), Folge sinkender temperaturen $ t_{1}, t_{2}, ... $, Akzeptanzbedingung:\\
$ c({\bf p}) > c({\bf q}) \ \ OR \ \ exp(\dfrac{c({\bf p}) - c({\bf q})}{t_{i}}) > Zuffalszahl \in [0,1] $, nächste Temperatur immer wenn akzeptiert wurde.

\subsection{Schwellwert-Algorithmus}
Akzeptanzbedingung: $ c({\bf p}) > c({\bf q}) - \sigma $, wobei $ \sigma $ immer weiter abgesenkt wird.

\subsection{Sintflut-Maximierung}
Akzeptanzbedingung: $ c({\bf p}) > F $, mit steigender unterer grenze F.

\subsection{Rekordjagd-Algorithmus}
Akzeptanzbedingung: $ c({\bf p}) \geq Rekord - \sigma $, bisher bester gesehener Wert, darf nicht um sinkende Toleranz unterschritten werden.

\section{Evolutionäre Algorithmen}
\subsection{Allgemein}
Population aus Individuen mit Merkmalsvektor \textbf{q} und Fitness c(\textbf{q}). Außerdem gibt es noch einen globalen Streuungsvektor $ {\bf \sigma} $, aus dem normal oder geometrisch verteilt mutiert wird. Individuen koennen sich fortpflanzen, bei mehreren Eltern Kreuzung, sonst Klon. Populationsgräße wird einigermaßen konstant gehalten, d.h. es werden immer wieder Loesungen verworfen.

\subsection{Plus-Evolutionsstrategie}
Erzeuge $ \lambda $ Nachkommen, nur die $ \mu $ fittesten Individuen aus den $ \mu $ Eltern und $ \lambda $ Nachkommen ueberleben.\\
Eltern werden pro Kind zufällig aus Population gewählt.

\subsection{Komma-ES}
Wie bei plus, allerdings sterben alle Eltern garantiert und nur $ \mu $ fittesten Kinder ueberleben.

\subsection{Klonen vs. Mehrere Eltern}
Beim Klonen mutiere einfach mithilfe von $ \sigma $ (s.o.).\\
Bei mehreren Eltern:
\begin{itemize}
\item Mischen: Es wird fuer jedes Merkmal ($ q_{i} $) zufaellig bestimmt, von welchem Elternteil dieses Merkmal uebernommen wird.
\item Mitteln: Es wird fuer jedes Merkmal der Mittelwert ueber die Werte der Eltern gebildet.
\end{itemize}

\section{Genetische Algorithmen}
Binäre Merkmalsvektoren, nur Nachkommen ueberleben.\\
Individuum klont sich mit Wahrscheinlichkeit $ W({\bf q}) = c({\bf q}) / \sum_{{\bf p} \in Pop} c({\bf p}) $. $ \mu $ Eltern erzeugen immer $ \mu $ Klon-Nachkommen.\\
Wähle unter den $ \mu $ Klonen p\% Individuen, die gekreuzt werden. Wähle daraus zufällige Paare. Aus a = (a1,...,an) und b = (b1,...,bn) entstehen c = (a1,...,aj,bj+1,...,bn) und d = (b1,...,bj,aj+1,...,an).\\
Dann kippe jedes Bit des Mermalsvektors mit sehr geringer Wahrscheinlichkeit.\\
Zusammengehörende Mermalsbits sollten möglichst nahe beisammen stehen, damit sie bei der Kreuzung nicht zerbrechen.

\section{Partikel-Schwarm-Optimierung}
Partikel jeweils mit Position $ p_{i}(t) $, Fitness $ f_{i}(p_{i}) $, Geschwindigkeit $ v_{i}(t) = p_{i}(t) - p_{i}(t-1) $.\\
Algorithmus:\\
Initiale Position und Geschwindigkeit zufällig. Dann sei $ q_{i} $ die bisher beste Position eines Partikels, q die global bisher beste Position.\\
Dann nächste Werte:\\
$ v_{i} = v_{i} * \omega + (q_{i} - p_{i}) * c_{1} * r_{1} + (q - p_{i}) * c_{2} * r_{2} $ r jeweils aus [0,1], c und $ \omega $ konstant.\\
Die Konstanten können per Superschwarm optimiert werden.

\section{Ameisen-Systeme}
Formeln abschreiben %TODO

\end{document}