\documentclass[8pt, DIV15, twocolumn]{scrartcl}
\usepackage[utf8]{inputenc}
\usepackage[T1]{fontenc}
\usepackage{amsfonts}
\usepackage[german]{babel}
\usepackage{amsmath}
\usepackage{caption}
\usepackage{float} 
\usepackage{color}
\usepackage{bm}
\usepackage{listings}

\usepackage{graphicx}

\usepackage{ae}
\subject{\vspace{-1\baselineskip}}

\title{Chapter 9 - De Casteljau-Algorithmus}
\date{}

\publishers{\vspace{-.5\baselineskip}}

\begin{document}
\setlength{\abovedisplayskip}{0pt}
\setlength{\belowdisplayskip}{0pt}
\setlength{\parskip}{0pt}
\setlength{\topmargin}{0pt}

 
\maketitle

\thispagestyle{empty}

\section*{Schreibweise}

$C$ ist De-Casteljau, $C^*$ Unterteilungsoperator mit Stufe $k$.

\begin{equation*}
\begin{aligned}
C \left( B_0^0, t \right) &= B_0^1 B_1^1 \\
C^* \left( B_0^0, k \right) &:= B_0^k ... B_{2^k - 1}^k 
\end{aligned}
\end{equation*}

\section*{Algorithmus}

\begin{lstlisting}[mathescape=true]
Eingabe:
	$\mathbf{b}_0^0 ... \mathbf{b}_n^0 \subset \mathbb{R}^d$
	$t \in \mathbb{R}$
Ausgabe:
	$\mathbf{b}_0^0 ... \mathbf{b}_0^n \; \mathbf{b}_0^n ... \mathbf{b}_n^0 \subset \mathbb{R}^d$

For k = 1, ..., n
	For i = 0, ..., n - k
		$\mathbf{b}_i^k \leftarrow \mathbf{b}_i^{k-1} \left( 1-t \right) + \mathbf{b}_{i+1}^{k-1} t$
\end{lstlisting}


\section*{Kanonische Parametrisierung}
Unterteilung eines Polygons in $P^k := \mathbf{P}_0^k ... \mathbf{P}_{2^k - 1}^k$ mit 

\begin{equation*}
\begin{aligned}
P_j^k &:= \mathbf{p}_0^{k,j} ... p_n^{k,j} \\
\mathbf{p}_i^{k,j} &:= 
\begin{bmatrix}
\beta_i^{k,j} \\
\mathbf{b}_i^{k,j}
\end{bmatrix} \\
\beta_i^{k,j} &:= \frac{j + \frac{i}{n}}{2^k}
\end{aligned}
\end{equation*}

Zu $P^k$ it die stückweise lineare Funktion $L_k \left( \mathbf{b}_0 ... \mathbf{b}_n \right)$

\section*{Differenzenpolygon}

\begin{equation*}
\begin{aligned}
\Delta B &:= \Delta \mathbf{b}_0 ... \mathbf{b}_n \\
&:= \left(\Delta \mathbf{b}_0 \right) ... \left( \Delta \mathbf{b}_{n-1} \right) \\
&:= \left( \mathbf{b}_1 - \mathbf{b}_0 \right) ... \left( \mathbf{b}_n - \mathbf{b}_{n-1} \right)
\end{aligned}
\end{equation*}

Zusammenhang von Differenzen mit normalem Polygon

\begin{equation*}
\begin{aligned}
B_0 B_1 &:= C^* \left(B, 1 \right) \\
\Delta B_0 \Delta B_1 &= \frac{1}{2} C^* \left(\Delta B, 1\right) \\
B_0 ... B_{2^k} &:= C^* \left(B, k \right) \\
\Delta B_0 ... \Delta B_{2^k} &= 2^{-k} C^* \left(\Delta B, k \right)
\end{aligned}
\end{equation*}

Abhängigkeit von $\Delta B_0$ und $\Delta B_1$ bzgl. C: (Übung 9.2)

\begin{equation*}
\begin{aligned}
&tL = \Delta B_0 \;\;\;
\left( 1 - t \right) R = \Delta B_1 \\
&\Rightarrow \frac{\Delta B_0}{t} \frac{\Delta B_1}{\left( 1 - t \right)} = C\left(\Delta B, t\right)
\end{aligned}
\end{equation*}

\section*{Ableitung unterteilter Polygone}

\begin{equation*}
\begin{aligned}
b \left( t \right) &:= L_k \left( B \right) \\
\beta_i^{k,j} &:= \frac{jn + i}{n 2^k} \\
b_i^{k,j} &:= b^k \left( \beta_i^{k,j} \right) \\
\dot{b}^k \left( t \right) &= \frac{\Delta b_i^{k,j}}{\Delta \beta_i^{k,j}} = n 2^k \Delta b_i^{k,j}
\end{aligned}
\end{equation*}

\section*{Konvergenz}
Satz: Zu jedem Polygon $B = b_0 ... b_n$ gibt es ein Polygon $L_\infty \left( B \right)$ vom Grad $\leq n$, sodass

\begin{equation*}
\sup\limits_{[0,1]} |L_k \left( B \right) - L_\infty \left( B \right)| \in O \left( 2^{-k} \right)
\end{equation*}

Folgerung:

\begin{equation*}
\frac{d}{dt} L_\infty \left( B \right) = n L_\infty \left( \Delta B \right)
\end{equation*}

\end{document}