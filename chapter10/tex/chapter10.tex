\documentclass[8pt, DIV15, twocolumn]{scrartcl}
\usepackage[utf8]{inputenc}
\usepackage[T1]{fontenc}
\usepackage{amsfonts}
\usepackage[german]{babel}
\usepackage{amsmath}
\usepackage{caption}
\usepackage{float} 
\usepackage{color}
\usepackage{bm}
\usepackage{listings}

\usepackage{graphicx}

\usepackage{ae}
\subject{\vspace{-1\baselineskip}}

\title{Chapter 10 - Lane-Riesenfeld-Algorithmus}
\date{}

\publishers{\vspace{-.5\baselineskip}}

\begin{document}
\setlength{\abovedisplayskip}{0pt}
\setlength{\belowdisplayskip}{0pt}
\setlength{\parskip}{0pt}
\setlength{\topmargin}{0pt}

 
\maketitle

\thispagestyle{empty}

\section*{Allgemein}
Biinfinite Kontrollpolygone: $\mathbf{c}_\mathbb{Z} := \left( \mathbf{c}_i \right)_{i \in \mathbb{Z}} = \left( ... \mathbf{c}_{-1} \mathbf{c}_0 \mathbf{c}_1 ... \right)$

\section*{Algorithmus}

\begin{lstlisting}[mathescape=true]
Eingabe:
	$\mathbf{c}_\mathbb{Z}^0$ ein Polygon $\subset \mathbb{R}^d$
	$n \in \mathbb{N}_0$ der Grad
Ausgabe:
	$\mathbf{c}_\mathbb{Z}^m \subset \mathbb{R}^d$

For k = 1, ..., m
	For i $\in \mathbb{Z}$   //verdoppele
		$\mathbf{d}_{2i}^0 \leftarrow \mathbf{d}_{2i+1}^0 \leftarrow \mathbf{c}_i^{k-1}$
	For j = 1, ..., n
		For i $\in \mathbb{Z}$   //mittele
			$\mathbf{d}_i^j \leftarrow \left( \mathbf{d}_{i-1}^{j-1} + \mathbf{d}_i^{j-1} \right) \frac{1}{2}$
	For i $\in \mathbb{Z}$   //bennen um
		$\mathbf{c}_i^k \leftarrow \mathbf{d}_i^n$
\end{lstlisting}

Bem.: $\mathbf{d}_0^j$ ist immer der Punkt $\mathbf{d}_i^j$ mit dem kleinsten $i$, der von $\mathbf{c}_0^0$ beeinflusst wird.

\section*{Unterteilungsmatrizen}

\begin{equation*}
\begin{aligned}
D = 
\begin{bmatrix}
\ddots & & & \\
& 1 & & \\
& 1 & & \\ 
& & 1 & \\
& & 1 & \\
& & & \ddots
\end{bmatrix} \; \;
M = \frac{1}{2}
\begin{bmatrix}
\ddots & & & & & \\
& 1 & 1 & & & \\
& & 1 & 1 & & \\
& & & 1 & 1 & \\
& & & & & \ddots
\end{bmatrix} \\ 
U_n^m \mathbf{c} = \left( U_n \right)^m \mathbf{c} = \left( M^n D \right)^m \mathbf{c}
\end{aligned}
\end{equation*}

Allgemein enthält $U_n$ in den Spalten die Einträge $\alpha_i$, jeweils um zwei Zeilen versetzt:

\begin{equation*}
\alpha_i = \begin{cases}
        \frac{1}{2^n} \binom{n+1}{i}, & i = 0, ..., n+1\\
        0, & sonst
        \end{cases}
\end{equation*}

Unterteilungsgleichung:

\begin{equation*}
b_i = \sum\limits_{k\in \mathbb{Z}} \alpha_{i-2k} c_k, \; i\in \mathbb{Z}
\end{equation*}


\section*{Das Symbol}
Bsp. $\alpha$ eines stationären Unterteilungsalgorithmus:

\begin{equation*}
\begin{aligned}
\alpha \left( z \right) &:= \sum\limits_{j\in\mathbb{Z}} \alpha_j z^j \\
\alpha_n \left( z \right) &:= \frac{1}{2^n} \sum\limits_{i=0}^{n+1} \binom{n+1}{i} z^i = \frac{1}{2^n} \left( 1 + z \right)^{n+1}
\end{aligned}
\end{equation*}

\section*{Differenzenschema}

\begin{equation*}
\begin{aligned}
\text{Rückwärtsdifferenzen: }\nabla \mathbf{c} &:= \left( \nabla c_i \right)_{i \in \mathbb{Z}} \\
\nabla c_i &:= c_i - c_{i-1} \\
\text{Symbol: } v \left( z \right) &:= 1 - z \\
\nabla \mathbf{c} \left(z\right) &= v \left(z\right) c \left(z\right) \\
\text{Differenzenpolygone: } b_\nabla \left( z\right) &= \left(1-z\right)b\left(z\right) \\
&= \left(1-z\right) \alpha \left(z\right) \frac{c_\nabla \left(z^2\right)}{\left(1-z^2\right)} \\
&= \frac{\alpha\left(z\right)}{1+z} c_\nabla \left(z^2\right)
\end{aligned}
\end{equation*}

Bem.: Das Differenzenschema zu $\alpha\left(z\right)$ existiert nur, wenn gilt:
$\alpha\left(-1\right) = \sum\limits_{i\in\mathbb{Z}} \alpha_{2i} - \sum\limits_{i\in\mathbb{Z}} \alpha_{2i + 1} = 0$

Bem.: Es muss $\sum\limits_{i\in\mathbb{Z}} \alpha_{2i} = \sum\limits_{i\in\mathbb{Z}} \alpha_{2i+1} = 1$ gelten, damit die unterteilten Polygone gegen eine Kurve konvergieren.

\section*{Uniforme Parametrisierung}
Brauchen wir das? Ich würde es eher weglassen. Verstehn wir ja eh nicht.
\section*{Konvergenz}
Same thing here.
\end{document}