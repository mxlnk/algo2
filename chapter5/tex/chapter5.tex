\documentclass[8pt, DIV15, twocolumn]{scrartcl}
\usepackage[utf8]{inputenc}
\usepackage[T1]{fontenc}
\usepackage{amsfonts}
\usepackage[german]{babel}
\usepackage{amsmath}
\usepackage{caption}
\usepackage{float} 
\usepackage{color}
\usepackage{bm}
\usepackage{listings}

\usepackage{graphicx}

\usepackage{ae}
\subject{\vspace{-1\baselineskip}}

\title{Chapter 5 - Geometrische Algorithmen}
\date{}

\publishers{\vspace{-.5\baselineskip}}

\begin{document}
\setlength{\abovedisplayskip}{0pt}
\setlength{\belowdisplayskip}{0pt}
\setlength{\parskip}{0pt}
\setlength{\topmargin}{0pt}

 
\maketitle

\thispagestyle{empty}

\section*{BRZ}

\begin{lstlisting}[mathescape=true]
Eingabe:
	Ein Polyeder $P \subset \mathbb{R}^3$
	orientierte, planare, disjunkte 
	Polygone $P_1,..., P_n \subset\mathbb{R}^3$
Ausgabe:
	Ein RZB fuer $P_1,...,P_n$

	k $\leftarrow$ 1
	for i = 1,...,n
	$Q_k \leftarrow P_i \cap P$
	falls $Q_k \neq \emptyset$
		k $\leftarrow$ k + 1
l $\leftarrow$ 1
falls $\exists j: Q_j$ zerlegt $P$
	l $\leftarrow$ j
Wurzel $leftarrow Q_l$
	$\vdots$ // rekursiver Aufruf mit Haelften
\end{lstlisting}

\section*{Konvexe Hüllen}

\begin{equation*}
\begin{aligned}
&M \subset A \text{ heißt konvex }:\Leftrightarrow \\
&\forall \mathbf{a}, \mathbf{b} \in M \forall t \in [0,1]: \\
&\mathbf{x} := \mathbf{a} \left(1-t\right) + \mathbf{b} t \in M
\end{aligned}
\end{equation*}

Def.: $[M] := $ konv $M$ = konv Hülle $M$

\section*{Dualität}

\begin{equation*}
\begin{aligned}
\mathbf{u}^t \mathbf{x} &= 1, \; \mathbf{u} \in \mathbb{R}^n \\
\mathbf{u}* &:= \{\mathbf{x}\in A| \mathbf{u}^t\mathbf{x} = 1\} \text{ heißt Hyperebene}
\end{aligned}
\end{equation*}

$\mathbf{u}\in A$ wird der zu $\mathbf{u}*$ \emph{duale Punkt} genannt. $\mathbf{x}*$ die zum Punkt $\mathbf{x}$ \emph{duale HE}.

$A* := \{\mathbf{x}*| \mathbf{x} \in A\}$ ist der \emph{Dualraum} zu $A$.
$\mathbf{u}^t\mathbf{x} \leq 1, \; \mathbf{u} \neq 0$ bildet den \emph{Halbraum} $\mathbf{u}^\leq$.

Satz.: $[[M]] := \text{ closure } [M]$

\begin{equation*}
\begin{aligned}
M_p &:= \{\mathbf{u}| M \subset \mathbf{u}^\leq\} \text{ Polarmenge von }M \\
M_{pp} &= \{\mathbf{x}| M_p \subset \mathbf{x}^\leq\} = [[M]]
\end{aligned}
\end{equation*}

\section*{Bez. zw. Knoten, Kanten, Facetten}

Für ein Polyeder mit $v$ Knoten, $e$ Kanten und $f$ Seiten gibt \emph{Eulers Formel}

\begin{equation*}
\begin{aligned}
&v - e + f = 2 \\
\text{Außerdem } f &= 2v - k - 4 = O\left(v\right) \text{wenn Polyeder trianguliert}
\end{aligned}
\end{equation*}

\end{document}