\documentclass[8pt, DIV15, twocolumn]{scrartcl}
\usepackage[utf8]{inputenc}
\usepackage[T1]{fontenc}
\usepackage{amsfonts}
\usepackage[german]{babel}
\usepackage{amsmath}
\usepackage{caption}
\usepackage{float} 
\usepackage{color}
\usepackage{bm}
\usepackage{listings}

\usepackage{graphicx}

\usepackage{ae}
\subject{\vspace{-1\baselineskip}}

\title{Chapter 12 - Textsuche}
\date{}

\publishers{\vspace{-.5\baselineskip}}

\begin{document}
\setlength{\abovedisplayskip}{0pt}
\setlength{\belowdisplayskip}{0pt}
\setlength{\parskip}{0pt}
\setlength{\topmargin}{0pt}

 
\maketitle

\thispagestyle{empty}


\section*{Naive Suche}

\begin{lstlisting}[mathescape=true]
Eingabe:
	t = $t_1 ... t_n$
	s = $s_1 ... s_m$ // Suchtext
Ausgabe:
	kleinstes $i$ mit $t_{i+1} ... t_{i+m}$ = s

i <- 0
Solange i $\leq$ n - m
	// vgl. s mit $t_{i+1} ... t_{i+m}$
	j <- m
	Solange j > 0 und $s_j = t_{i+j}$
		j <- j - 1
	Falls j = 0
		gib $i$ aus; stop
	sonst i <- i + 1 
\end{lstlisting}

\section*{Vorkommens-Heuristik}

Vorkommensfunktion:

\begin{equation*}
\begin{aligned}
	v : A &\rightarrow {0, ..., m} \\
		a &\mapsto v(a), \\
	v(a) &:= min \{\;k| a \not \in s_{k+1} ... s_m \;und\; \left( a = s_k \vee k = 0 \right) \}
\end{aligned}
\end{equation*}

\subsection*{Bem.}
Kann vorberechnet werden in $O(|A| + m)$.

\subsection*{Änderungen in "`Naive Suche"'}
Sei $t_{i+j} \neq s_j$ und $t_{i+j+1} ... t_{i+m} = s_{j+1} ... s_m$.

Dann ist $v := v(t_{i+j}) \neq j$ und

\begin{itemize}
	\item falls $v < j$, kann $i$ um $j-v$ erhöht werden
	\item falls $v > j$, kann $i$ um $m-v+1$ erhöht werden
\end{itemize}

\subsection*{Bem.}
Für kleines $m$ und großes Alphabet $A$ nun Laufzeit $O(\frac{n}{m})$


\section*{Suffixfunktion}

\subsection*{Def.}
Ein Präfix eines Wortes $w$, das zugleich ein Suffix von $w$ ist, heißt \emph{Präsuffix}. $geps(w)$ ist das größte echte Präsuffix von $w$.

$w_j := s_{j+i} ... s_m$.

\begin{equation*}
\gamma \left( j \right) := |geps \left( w_j \right)|
\end{equation*}

Suffixfunkion $\sigma$ wird dargestellt:

\begin{equation*}
\sigma \left( j \right) = min \left( \{k \in \{1, ..., j\}| \; \gamma \left(j-k \right) = m-j\} \cup \{m - \gamma \left(0 \right)\} \right)
\end{equation*}

\subsection*{Algorithmus für $\sigma$ (Laufzeit $O\left(m\right)$)}

\begin{lstlisting}[mathescape=true]
Eingabe:
	$\gamma$(0 ... m)
Ausgabe:
	$\sigma$(1 ... m)

For j = 1, ..., m
	$\sigma$(j) $\leftarrow$ m - $gamma$(0)
For i = 0, ..., m-1
	k $\leftarrow$ m - $\gamma$(i) - i
	j $\leftarrow$ m - $\gamma$(i)
	// Es gilt jetzt $\gamma$(j-k) = m - j
	falls $\sigma$(j) > k
		dann $\sigma$(j) $\leftarrow$ k
\end{lstlisting}

\subsection*{Algorithmus für $\gamma$ (Laufzeit $O\left(m\right)$)}

\begin{lstlisting}[mathescape=true]
Eingabe:
	$s_1 ... s_m$
Ausgabe:
	$\gamma$(0 ... m-1)

$\gamma$(m-1) $\leftarrow$ 0
For i = m-1, ..., 1
	j $\leftarrow$ m - $\gamma$(i)
	Solange $s_i \neq s_j$ und m $\neq$ j
		j $\leftarrow$ m - $\gamma$(j)
	falls $s_i = s_j$
	dann $\gamma$(i-1) $\leftarrow$ m-j+1
	sonst $\gamma$(i-1) $\leftarrow$ 0
\end{lstlisting}

\section*{Boyer-Moore-Algorithmus}

\begin{lstlisting}[mathescape=true]
// wie in "Naiver Suche", letzte Zeile
// durch folgendes ersetzt
v $\leftarrow$ v($t_{i+j}$)
falls v < j
	i $\leftarrow$ i + max{j-v, $\sigma$(j)}
sonst
	i $\leftarrow$ i + max{m-v+1, $\sigma$(j)}
\end{lstlisting}

\end{document}